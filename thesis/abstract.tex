
\begin{abstract}


数据立方(Data Cube)是一种有效支持 OLAP 的多维数据计算模型。它通过预先计算数据表中各属性间所有组合对应的 GroupBy 结果并将其存储起来,以缩短系统的响应时间从而提高查询效率。随着数据量的急剧增长,分布式计算(如MapReduce)的使用日益广泛,将数据立方计算与分布式结合是必然的趋势。

对于代数度量,如 SUM 等,简单地采用 MapReduce 框架即可高效地完成数据立方的计算。但对于整体性度量,如DISTINCT等,若与MapReduce 简单地结合,则会出现负载不均衡、中间数据过多等问题。当前最好的分布式数据立方计算算法MR-Cube,通过数据划分、合并计算的方法减缓上述问题。但是该算法对数据划分不够精准,会导致一些不必要的数据划分,加重之后的合并操作。而对于合并计算,该算法仅提出了一些规则,而无简单且有效的合并方法,并且进行合并计算时使用 BUC 算法亦未充分利用 MapReduce 框架的特性。

为了更好地解决负载不均衡、中间数据过多的问题,本论文借鉴TeraSort与PipeSort,提出TSP-Cube算法。TSP-Cube 借鉴 TeraSort 随机抽样的思想,根据数据出现的频率对数据进行划分,不仅可以有效避免不必要的划分,并且适用于各种分布类型的数据集,从而有效解决负载不均衡的问题。同时TSP-Cube采用能充分利用 MapReduce 框架特性的 PipeSort 替代 MR-Cube 中的 BUC 进行合并计算,并且针对层次型的数据集,根据其属性特征以及PipeSort的特性,采用更简单有效且均匀的合并计算方案,从而解决中间数据过多的问题。

论文通过实验证明,无论在均匀分布或是倾斜分布下,TSP-Cube 在整体性度量函数中都有更好的性能,比已有的分布式算法更通用。此外,实验还对多种算法在代数度量下的性能进行了比较,从而得出不同类型的度量应采用的方法。

\keyword{数据立方,分布式,MapReduce,TeraSort}
\end{abstract}


\begin{abstract}[english]


Data cube is a multidimensional data model which effectively supports OLAP. It can reduce the responding time of queries and improve the efficiency of application via pre-calculating and storing results of GroupBys of all combinations of attributes. As the explosion of massive data, it is an inevitable trend of the integration of data cube materialization and distributed computing model (e.g. MapReduce), which has been more and more widely used.

Data cube materialization can be efficiently completed by simply using the framework of MapReduce for algebraic measures, e.g. SUM. While for holistic measures, such as DISTINCT, if we just integrate with MapReduce as the way algebraic measures do, it will lead to the problems of load imbalance and tons of intermediate data. The state-of-the-art distributed algorithm MR-Cube try to alleviate these two problems via data partitioning and batch area calculation. However, MR-Cube is not accurate for data partitioning and may still lead to load imbalance under extremely skewed circumstance. In batch area calculation, MR-Cube only propose some rules rather than a simple and specific batch method, and the algorithm to calculate GroupBys is BUC, which cannot make full use of the performance of the framework of MapReduce. 

In this paper, we propose TSP-Cube which borrows ideas from TeraSort and PipeSort, in order to thoroughly solve the problem of load imbalance and tons of intermediate data. Borrowing from the idea of random sample of TeraSort, TSP-Cube partitions the data according to the frequencies of data in sampling, which not only reduces or even avoids unnecessary data partitioning, but also be suitable for different types of distribution. Meanwhile TSP-Cube applies Pipesort instead of BUC for batch area calculation, because PipeSort can take fully advantages of the characteristic of the framework of MapReduce. In addition, for the specific hierarchical data set, TSP-Cube puts forward a pipeline generation method for the generation of batch area according to the features of attributes of data sets and the characteristic of PipeSort, and then solves the problem of tons of intermediate data.  

Finally, we demonstrate that, TSP-Cube has better performance and more general in cube materialization with holistic measures, compared with current state-of-the-art algorithms, no matter under uniform distribution or extreme skewed distribution. The experiment also includes a comparison of algebraic measures, and then we can give a conclusion about the best algorithms under different situations.


\keywordenglish{Data Cube,Distribution,MapReduce,TeraSort}
\end{abstract}




%数据立方 (Data Cube) 是一种有效支持 OLAP 的多维数据计算模型。它通过预先计算数据表中各属性间所有组合对应的 GroupBy 结果并存储起来,以缩短系统的查询响应时间从而提高应用效率。随着数据量的急剧增长,分布式计算 (如MapReduce) 的使用越来越广泛。因此将数据立方计算与分布式结合是必然的趋势。目前 MapReduce 与数据立方在代数度量函数下,如 SUM 等,已经有较好的结合。但对于整体性度量函数,如 DISTINCT 等,若与 MapReduce 简单地结合,会出现负载不均衡、中间数据过多等问题,从而对性能有较大的影响。尽管已经有论文提出 MR-Cube 算法,通过数据划分、合并计算的方法解决上述问题,但是该算法对数据划分并不够精准,会产生一些不必要的数据划分,加重之后的合并操作。而对于合并计算,该算法仅提出了一些规则,而无简单且有效的合并方法。并且该算法使用 BUC 进行合并计算,并未充分利用 MapReduce 框架的特性。本论文借鉴 TeraSort 与 PipeSort,提出 TSP-Cube。TSP-Cube 根据 TeraSort的思想,改进了数据的划分方式。它根据数据出现的频率,准确地定位需要划分的数据,减少甚至避免不必要的划分,并且无论在一般或者倾斜的数据集下,都能均匀地划分数据。同时 TSP-Cube 提出使用 PipeSort 替代已有论文中的 BUC 方法计算数据立方,充分利用 MapReduce 框架的特性。TSP-Cube 还针对层次型的数据集,根据其属性特征以及 PipeSort 的特性,提z更简单有效且均匀的合并计算方案,从而解决中间数据过多的问题。最后论文通过实验比较 TSP-Cube 与多种算法在不同的数据分布下性能、负载等方面的差别,从而验证无论在均匀分布或是倾斜分布下,TSP-Cube 在整体性度量函数计算中也能有较好的性能,比已有的方法更具有通用性。实验中还包括了多种算法在代数度量函数下的比较,从而分析不同类型的度量应使用的方法。