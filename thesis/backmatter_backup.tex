
\chapter{致谢}

这篇论文能够完成至此,首先非常感谢我的家人一直以来的支持,正因为他们长期的鼓励给予了我非常大的信心与动力,尤其在我自我怀疑时,他们的鼓励让我相信我是可以完成这份毕业设计的。

在这里,我要非常感谢我的导师冯剑琳教授。他一直给予了非常多的指导,无论是在学术上,还是在为人处事上。由于我选择了数据库作为研究方向,从大四开始,就接触了一些与数据库相关的内容。在研一时,接触到了索引,在老师的鼓励下,阅读了BerkeleyDB BTree部分的代码,同时由于TA的工作,正好将这份源码与课程设计相结合,加深了自己对源码的理解,这个阅读源码的经历对我而言受益匪浅。这些过程中冯老师都给予了很多的指导,引导我如何思考问题,批判性地阅读论文,而不是被论文牵着鼻子走。

在毕业论文开题时,我的方向转到了数据立方上,这对我而言是一个崭新的课题,也是一个很大的挑战。一开始我很怀疑自己能否完成,甚至想是不是放弃这个题目算了。但是我希望自己坚持,哪怕最后弄砸了。这期间冯老师给了我很多引导,也指出了我论文不足的地方,对我的论文有非常大的帮助。老师一直跟我们说,做学术研究不一定会成功,重要的是体会这个过程。论文写到此,对这个研究过程也总算有些许体会。

与此同时,还要在此感谢黄强同学与翁镇斌同学,黄强同学对我的毕业设计给予了非常多的宝贵意见。翁镇斌同学对分布式的深入了解给予了我非常多的技术支持。

最后,感谢我的母校中山大学,感谢软件学院,感谢实验室的每一位同学,感谢我的室友,感谢曾经帮助过我的每一位老师和同学。

谨以此文献给所有关心、支持、鼓励过我的亲人、师长和朋友。\nopagebreak
