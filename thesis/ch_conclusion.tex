
\chapter{总结与展望}

数据立方是 OLAP 领域内一项关键技术,在现实应用中,数据立方的高效计算是实例化数据立方的关键。数据立方的高效计算可大幅度提高 OLAP 对查询的相应时间。分布式计算随着数据量爆炸式地增长,被越来越广泛地使用。因此将数据立方的计算与分布式结合是必然的趋势。数据立方的计算效率不仅仅与数据量有关,还与度量函数、数据分布等因素有关,因此数据立方的计算并不是简单地能用一种方法解决。尤其当它与分布式框架MapReduce结合时,可能带来更多的问题。

度量函数主要划分为两类,代数度量与整体性度量。对于代数度量函数,数据立方的计算相对更为简单,由于数据的可划分性与中间结果的确定性,它与MapReduce能很好地结合。但对于整体性度量函数,问题就变得复杂了。尤其当它与MapReduce结合时,由于数据的不可划分,在倾斜的数据集下,直接使用代数度量函数的方法来计算整体性度量函数,会给MapReduce带来负载不均衡的问题。MR-Cube 的提出正是解决在倾斜数据集下整体性度量函数的计算问题。它提出了部分代数度量的方式使整体性度量函数也能根据特定的属性值进行划分,并且通过采样的方式确定数据集中是否有大group,从而对该region内所有的group进行划分,减少MapReduce负载不均衡的问题。同时它还提出了 BatchArea 的概念,将多个 group 放在同一个 reduce 函数内计算,减少中间数据的产生。

MR-Cube 虽然实现了数据立方与 MapReduce 的高效结合,但其仍存在缺陷,尤其是在极端倾斜的数据集下,会产生许多不必要的数据划分,而它对数据划分的方法也很可能退化失效。同时它对 batch 的划分只给出了一些建议遵循的规则,并没有给出具体的简单有效的划分方法。针对以上不足,本文提出了一种TSP-Cube的计算方法。TSP-Cube对MR-Cube的改进主要分为两个方面。一个方面是改进了MR-Cube对数据的划分,借鉴TeraSort的思想对数据进行划分,可避免不必要的数据划分,并且即使在极端倾斜的数据分布下,也能对数据进行均匀的划分,比MR-Cube对数据的划分更有通用性。另一方面,对 BatchArea 提出了更为直接简单并且有效的生成方法,尤其针对层次型的数据给出了均匀且有效的生成 batch 的方法。同时,利用MapReduce的特性,提出使用 Pipesort 替换 BUC 计算batch内多个 group 的聚合。

论文通过实验说明在不同的度量函数,不同的数据分布下,多种方法的差异,从而说明没有一种方法是可以解决所有的问题。通过实验结果可看出,对于代数度量函数,由于MapReduce框架的特性,使用Naive的方法足以应对计算。但当处理整体性度量函数的计算,尤其当数据具有倾斜性时,TSP-Cube比MR-Cube更具有通用性,可解决一般倾斜至极端倾斜的情况。

在未来的工作中,数据立方仍有许多需要探索的地方。一方面,整体性度量是更为复杂的度量,并且可能有嵌套的情况,这令数据划分变得更为复杂,因此需要研究总结出对数据划分更为通用的方法。另一方面,随着维度数量的增加,数据立方的大小呈指数型增长,因此部分数据立方的计算在这种情况下更有研究价值。甚至可以考虑使用采样,通过概率统计等方法估算数据立方的结果,这是一种为了实时性而牺牲精度的方法。最后,MapReduce只是分布式框架中的一种,之后的研究可以考虑是否有其他分布式框架更适合与数据立方的计算相结合。
