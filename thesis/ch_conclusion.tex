
\chapter{结语}

数据立方是 OLAP 领域内一项关键技术,数据立方的高效计算可大幅度提高 OLAP 对查询的相应时间。分布式计算随着数据量爆炸式地增长,被越来越广泛地使用。因此将两者结合是必然的趋势。


联机分析处理(OLAP) \cite{chaudhuri1997overview} 是一种从多维度分析数据的商业智能分析技术,能方便的进行大规模数据分析及统计计算,多用于决策支持系统和数据仓库。例如企业可从用户维度,产品维度以及订单维度来分析对于不同年龄阶层最受欢迎的产品类型,从而可推出针对不同年龄的产品销售计划。由于企业数据的快速累积增长以及企业对分析历史数据、挖掘商业价值的热情逐渐增高,联机分析处理应用在业界变得越来越受欢迎,开发更高效的 OLAP 技术浪潮应运而生。全球最大的商业社交网站 LinkedIn 的网页分析工具 Avatara \cite{wu2012avatara} 正是基于 OLAP 技术开发的。关系型联机分析处理 (ROLAP) \cite{pedersen2001multidimensional}是 OLAP 其中一种较为流行的实现技术,它针对存储于关系型数据库(RDMS)中的数据进行动态的多维分析。相对于同样是较为流行的 OLAP 实现技术,还有多维联机分析处理(MOLAP)。ROLAP 由于能够直接访问存储于 RDMS 中的数据,所以不需要对分析数据进行预处理计算和额外存储,且 ROLAP 还能基于 RDMS 执行 SQL 查询来满足用户请求,所以基于这几点,ROLAP 相对于 MOLAP 能够支持更多的用户群组及数据量。如全球最大的独立 BI 公司 MicroStrategy,其推出的 OLAP 工具 \cite{msolapservice} 正是以 ROLAP 技术作为其主要的分析架构。

数据立方(Data cube) \cite{gray1997data}是由 Jim Gray 等人于 1996 年提出的,一种为有效支持 OLAP 应用的多维数据存储模型,是 OLAP 领域中的一项关键技术。通过预先计算数据表中各属性间的所有组合对应的 GROUP BY 结果并存储起来,以缩短决策支持系统的查询响应时间提高应用效率。在 OLAP 术语中,聚合属性称为维属性,被合计的属性称为度量属性,在数据立方中,一个特定的 GROUP BY 称为一个数据小方 (cuboid)。在现实应用中,数据立方的高效计算是实例化数据立方的关键。广泛流行的数据立方计算方法主要是基于单机实现的 \cite{agarwal1996computation} \cite{beyer1999bottom},而对于并行数据立方计算的研究则主要是局限在小集群上 \cite{ng2001iceberg} \cite{dehne2002parallelizing}。随着数据量的急剧增长,业界开始寻求在分布式计算框架下进行高效的数据立方实例化。如前雅虎搜索与运计算首席科学家,现微软技术会士 Raghu Ramakrishnan 则带领其团队进行 Data Cube 与 MapReduce 的技术整合,提出了 MR-CUBE 方案 \cite{nandi2012data} \cite{nandi2011distributed},针对 MapReduce 框架提出了 Data Cube 的实例化计算方法,并提出了 2 种 Holistic 方法在分布式计算下的解决方案。MR-CUBE 的思想现也在 Pig 开源项目上进行实现 \cite{mrcubepig}。



随着数据的爆炸性增长,传统的对数据处理计算分析的方法已无法满足当前的需求。仅仅将传统算法与分布式机械地结合,并不能充分利用两者各自的优势。\cite{cuzzocrea2011analytics} \cite{cuzzocrea2013data} \cite{cuzzocrea2013big} 中提出了一些由于数据增长给数据仓库、OLAP 带来的挑战,其中包括大数据的存储与分布、可扩展问题、ETL过程、多维数据的建模、分析、查询等等。

数据立方是 OLAP 领域内一项关键技术,数据立方的高效计算可大幅度提高 OLAP 对查询的相应时间。业界一直在从多个角度探讨基于数据立方技术的优化 \cite{xin2003star} \cite{harinarayan1996implementing} \cite{zhao1997array} \cite{han2001efficient} \cite{wang2002condensed}。随着数据的急剧增加,探讨如何在分布式计算框架下高效的完成数据立方的计算成为势在必行的一个趋势 \cite{abello2011building} \cite{wang2010mapreducemerge} \cite{sergey2009applying} \cite{lee2012efficient} \cite{wang2013scalable}。

Raghu Ramakrishnan 团队提出的 MapReduce DataCube 方案,以下简称 MR-CUBE, \cite{nandi2012data} \cite{nandi2011distributed},虽然实现了数据立方与 MapReduce 的高效结合,但其仍存在缺陷,尤其是在特别倾斜的数据集下,其对数据划分的方法很可能退化失效。同时它对维度的划分只给出了一些建议遵循的规则,并没有给出具体的简单有效的划分方法。此外,MR-CUBE 使用了一种基于 BUC \cite{beyer1999bottom} 的算法计算 MapReduce 下的数据立方。而基于 ROLAP 的数据立方实现技术除了 BUC 外,还有 PipeSort,PipeHash \cite{agarwal1996computation} 等。 如 Suan Lee 等人提出了基于 PipeSort 的 MapReduce Cube 实现方法 \cite{lee2012efficient} 。这些方法与 BUC 相比,更能利用 MapReduce 框架的一些优势。业界对于整合 Data Cube 与 MapReduce 的技术仍在探讨中,所以可尝试用其他 ROLAP Cube 的实现方法来探讨取代 BUC,从而优化 MR-CUBE、整合其他 ROLAP Cube 技术与 MapReduce。同时对于倾斜数据的均匀划分,可借鉴 \cite{tao2013minimal} 中的思想以及它对 Tera Sort 的使用。